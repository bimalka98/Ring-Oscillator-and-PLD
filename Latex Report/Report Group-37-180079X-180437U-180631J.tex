%  -----------------------------------------------------------------------------
%  Author         : Bimalka Piyaruwan Thalagala
%  GitHub         : https://github.com/bimalka98
%  Last Modified  : 07.05.2021
%  -----------------------------------------------------------------------------

\documentclass[a4paper,11pt]{article}%,twocolumn
\input{settings/packages}
\input{settings/page}
\input{settings/macros}
\usepackage{float}
\usepackage{circuitikz}


\begin{document}

\begin{titlepage}
\center % Center everything on the page

%-------------------------------------------------------------------------------------
%	HEADING SECTIONS
%------------------------------------------------------------------------------------
\textbf{\large Department of Electronic and Telecommunication Engineering}\\[0.5cm]
\textbf{\Large University of Moratuwa, Sri Lanka}\\[1cm]
\textbf{\large EN 2110 - Electronics - III}\\[2cm]
\includegraphics[width=0.3\textwidth]{figures/uomlogo}\\[2cm]

	
%-------------------------------------------------------------------------------------
%	TITLE SECTION
%------------------------------------------------------------------------------------
\textbf{\Huge Group Project - Group 37}\\[0.5cm]
\textbf{\Large Project Report}\\[5cm]


%----------------------------------------------------------------------------------------
%	MEMBERS SECTION
%----------------------------------------------------------------------------------------

\textbf{\large Submitted by}\\[0.5cm]
\begin{minipage}{0.2\textwidth}
	\begin{flushleft}	   
		{\large Caldera H. D. J.}\\[4mm]
		{\large Oshan J. W. P.}\\[4mm]
		{\large Thalagala B.P.}\\[4mm]
		
	\end{flushleft}
\end{minipage}
\hspace{5mm}
\begin{minipage}{0.2\textwidth}
	\begin{flushright}
		{\large 180079X}\\[4mm]
		{\large 180437U}\\[4mm]
		{\large 180631J }\\[4mm]
		
	\end{flushright}
\end{minipage}\\[1.5cm]

%----------------------------------------------------------------------------------------
%	DATE SECTION
%----------------------------------------------------------------------------------------
\textbf{\large Submitted on}\\[0.5cm]
\textbf{\Large \today} % Date, change the \today to a set date if you want to be precise

%----------------------------------------------------------------------------------------

\vfill % Fill the rest of the page with whitespace

\end{titlepage}

\tableofcontents
\vspace{4cm}
\begin{table}[H]
		\centering
		\begin{tabular}{l c l}
		\textbf{Name} & \textbf{Index} & \textbf{Contribution}\\

	Caldera H. D. J. &  180079X& PLD - Part 1, Part 2\\
	Oshan J. W. P.    & 180437U& PLD - Part 3 \\
	Thalagala B.P. & 180631J & Parasitic effect in Timing analysis \\


		\end{tabular}
		\caption{Contributions of each member}
\end{table}




\pagebreak
\section{Parasitic effect in Timing analysis}
\textbf{Objective}: \textit{Design a 3 stage (3 inverters) ring oscillator. Find the correlation of the parasitic effect in the oscillation period.}

\subsection{System Design}
Ring oscillator is a combination of delay stages arranged in series to form a closed loop chain. It consists of an \textbf{\textit{odd number of identical inverters (NOT gates)}} and it produces a  periodically oscillating output. The period of oscillation($T$) of a ring oscillator can be expressed as follows where $n$ is the number of cascaded NOT gates and $\tau_{d}$ is the propagation delay of a single stage\cite{rosci}. In this case $n = 3$.

\[
T = 2.n.\tau_{d} \hspace{5mm}\Longrightarrow ~ Oscillation~frequency = \frac{1}{2.n.\tau_{d}}
\]
\vspace{5mm}

Following figure illustrates the input output characteristic of an ideal inverter. There, output is changed as soon as the input signal changes. That is no propagation delay.

\begin{figure}[H]
	\centering
	\subfigure[Ideal Inverter Model]
	{ \includegraphics[scale=0.27]{figures/cct1plot5}
	}\hfill
	\subfigure[Ideal inverter Output]
	{ \includegraphics[scale=0.27]{figures/cct1plot6}
	}
\end{figure}

Whereas the below figure illustrates the input output characteristic of a real inverter. It can be observed that finite amount of time is required for output to be valid for a given valid input due to the time taken to charge and discharge the internal parasitic capacitors. This model has  been used to find the parasitic effect on the oscillation period as it is more realistic. For this experiment an NMOS and PMOS models were custom designed and Parameters of those models are as follows.

{\scriptsize
\begin{verbatim}

	.MODEL CMOSP PMOS KP=96u VTO=0.906 LAMBDA=0.01 TOX=21n GAMMA=0.486 CGDO=54p CGSO=54p CGBO=336p
	.MODEL CMOSN NMOS KP=96u VTO=0.786 LAMBDA=0.01 TOX=21n GAMMA=0.586 CGDO=402p CGSO=402p CGBO=362p
\end{verbatim}}

\begin{figure}[H]
	\centering
	\subfigure[Inverter Model with additional delay element]
	{ \includegraphics[scale=0.27]{figures/cct1plot3}
	}\hfill
	\subfigure[Effect of Parasitic Capacitance]
	{ \includegraphics[scale=0.27]{figures/cct1plot4}
	}
\end{figure}

\pagebreak
Basic structure of a ring oscillator can be depicted as follows using logic gates.

\begin{figure}[H]
	\centering
	\begin{circuitikz}
 \draw	 (1,0) node[ieeestd not port] (not1) {}
		 (3,0) node[ieeestd not port] (not2) {}
		 (5,0) node[ieeestd not port] (not3) {}
		 (0,0) -- (not1.in)
		 (not2.in) -- (not1.out)
		 (not3.in) -- (not2.out)
		 (not3.out) -- (6,0)
		 (not1.in) -- (0,0)
		 (6,0) -- (6,-1.5) -- (0,-1.5) -- (0,0)
		 ;
	\end{circuitikz}
	\caption{Basic structure of a 3 stage ring oscillator}
\end{figure}



A genral Ring Oscillator can be implemented using NMOS, PMOS transistors and additional delay elements(capacitors). But as the objective is to find the correlation between parasitic effect and oscillation period, the additional delay elements were not used and only the internal capacitance were taken in to account. Each inverter consists of one PMOS and one NMOS named as CMOSP and CMOSN.

\subsection{Simulation Results and Discussion}

\textit{\textbf{Note 1:} Only the gate-to source  capacitance were considered for the analysis, as it is what matters the most for the propagation delay of a MOSFET. In addition to that when changing the capacitor values, maximum gate-to source  capacitance was set to be 1 $\mu$F only to clearly illustrate the variation in the oscillation period. The real values are in pF range.}\\

\textit{\textbf{Note 2:} gate-to source  capacitance of the PMOS transistors were kept constant and only the NMOS transistors' gate-to source  capacitance were changed as it is sufficient for identifying the effect of parasitic capacitance on the period of oscillation.}\\

Following ring oscillator schematic has been used to simulate the effect of parasitic capacitance on the period of the output waveform. Simulation was run for $2~\mu s$ time intervals({\tt .tran 2u}) for 18 different gate-to source  capacitor values starting form 402 pF to 1 $\mu$F.\\

For the simulation to work an initial pulse is required. This requirement was satisfied through defining an initial condition({\tt .ic V(Vout) = 0}) for the {\tt Vout} node using LTSpice XVII's \textit{Spice Directive} feature. The same feature was used  to sweep\cite{sweeping} through different gate-to source  capacitor values. ({\tt .save V(Vout), .step dec param x 402p 1u 5})


\begin{figure}[H]
	\centering
	\includegraphics[scale=0.6]{figures/cct1plot1}
	\caption{3 stages CMOS Ring Oscillator}
\end{figure}
\pagebreak
When the gate-to source  capacitance is increased, as the plot depicts the period of oscillation increases and hence the frequency decreases. Because as mentioned at the beginning, the period of oscillation($T$) is affected by $\tau_{d}$, the propagation delay of a single stage which depends on the parasitic capacitance.
\[
Oscillation~frequency = \frac{1}{6.\tau_{d}}
\]

\begin{figure}[H]
	\centering
	\includegraphics[scale=0.5]{figures/cct1plot2}
	\caption{Waveform for Voltages of 3 stages CMOS Ring Oscillator}
\end{figure}

\subsection{Cross-section of a MOSFET}

Each MOSFET has 4 terminals called \textit{Body/Substrate (B), Source (S), Gate (G) and Drain (D)}. As following figure depicts the gate terminal is electrically insulated from the substrate semiconductor by a layer of oxide(shown in gray). This Metal-Oxide-Semiconductor structure forms a capacitor which is known as the \textbf{\textit{gate-to source capacitance}} This is what contributes to the propagation delay of the MOSFET \textit{\textbf{the most}} and hence to the period of oscillation of the Ring Oscillator.

\begin{figure}[H]
	\centering
	\includegraphics[scale=0.5]{figures/Cmos_impurity_profile.png}
	\caption{Cross-section of a MOSFET\cite{mosfet}}
\end{figure}

\pagebreak
\section{PLD}
\subsection{Part 1}
\textbf{Objective}: \textit{Design a programmable logic block to configure it as a `NAND' or a `NOR' gate using a single selection bit.}\\

First a truth table is drawn for this part considering a single selection bit (S) with two inputs (A, B) such that S=0 for ‘NAND’ and S=1 for ‘NOR’ operations respectively.
\begin{table}[H]
	\centering
	\begin{tabular}{|c |c| c| c|}
		\hline
		S & A & B & F \\\hline
		0 & 0 & 0 & 1 \\
		0 & 0 & 1 & 1 \\
		0 & 1 & 0 & 1 \\
		0 & 1 & 1 & 0 \\\hline
		1 & 0 & 0 & 1 \\
		1 & 0 & 1 & 0 \\
		1 & 1 & 0 & 0 \\
		1 & 1 & 1 & 0 \\\hline\hline
	\end{tabular}
\caption{The truth table}
\end{table}

Then the relevant logic expression was obtained using a karnaugh map and it was further simplified to obtain the combination of 'NAND' and 'NOR' operations.

\begin{table}[H]
	\centering
	\begin{tabular}{c |c| c| c| c}
		S\textbackslash AB & 00 & 01 & 11 & 10\\\hline
		0 & 1 & 1 & 0 & 1\\\hline
		1 & 1 & 0  &0  & 0
	\end{tabular}
	\caption{Karnough Map for the above truth table}
\end{table}

\[
\begin{split}
	F &= \overline{S}.\overline{A} + \overline{S}.\overline{B} + \overline{A}.\overline{B}\\
	&= \overline{S}.(\overline{A}+\overline{B}) + \overline{A}.\overline{B}\\
	&= \overline{S}.(\overline{A.B}) + \overline{A+B}\\
	&= \overline{S+A.B} + \overline{A+B}\\
	& = \overline{(S + A.B).(A+B)}\\
	& =\overline{(S + \overline{\overline{A.B}}).(\overline{\overline{A+B}})}
\end{split}
\]



So, the resultant combinational logic circuit is as follows. (2 NANDs, 2 NORs, 3 NOTs) For the implementation of this circuit; ‘NOT’, ‘NAND’, and 'NOR' gates were designed using 'NMOS' and 'PMOS' transistors. Their schematics in LTspice are depicted below.

\begin{figure}[H]
	\centering
	\subfigure[Schematic of NOT gate]
	{ \includegraphics[scale=0.18]{figures/2part1/NOT.pdf}
	}\hfill
	\subfigure[Schematic of NAND gate]
	{ \includegraphics[scale=0.18]{figures/2part1/NAND.pdf}
	}\hfill
	\subfigure[Schematic of NOR gate]
	{ \includegraphics[scale=0.18]{figures/2part1/NOR.pdf}
	}
\caption{Basic Gates in the CMOS logic}
\end{figure}
%
%
Finally, the PLD block is designed using the above gates and the waveforms were obtained.

\begin{figure}[H]
	\centering
	\includegraphics[scale=0.55]{figures/2part1/cct.pdf}
	\caption{Circuit designed using logic blocks}
\end{figure}

\begin{figure}[H]
	\centering
	\includegraphics[scale=0.5]{figures/2part1/block.pdf}
	\caption{Designed PLD block}
\end{figure}


\begin{figure}[H]
	\centering
	\includegraphics[scale=0.5]{figures/2part1/wave.pdf}
	\caption{Waveforms for inputs and output of PLD }
\end{figure}

\hrule
\subsection{Part 2}
\textbf{Objective}: \textit{Design a single switch matrix using six pass transistors.}\\

In this part, the single switch matrix is needed to be designed using six pass transistors. So as the first step, a pass transistor was designed and its performance was checked. Simply an NMOS transistor is fed with a switch could be used for this task.\\

For signal to travel in both directions ( from source to drain and from drain to source ) the NMOS has to be symmetric and to have the switch property, both gate-source and gate-drain of the NMOS should be reverse biased. So body terminal needs to be connected as the lowest voltage in the circuit. So we grounded the body terminal.\\

So when the switch is on, the input signal will be received at the output. But as we are trying to illustrate a real model, the voltage of the output signal will be somewhat reduced compared with the input signal.\\

For this task, an NMOS model was chosen and the defined parameters are as follows.

\begin{verbatim}
L=0.9u W=1.8u
.MODEL N_1u NMOS LEVEL = 3
+ TOX = 200E-10 NSUB = 1E17 GAMMA = 0.5
+ PHI = 0.7 VTO = 0.8 DELTA = 0.1
+ UO = 650 ETA = 3.0E-6 THETA = 0.1
+ KP = 120E-6 VMAX = 1E5 KAPPA = 0.3
+ RSH = 0 NFS = 1E12 TPG = 1
+ XJ = 500E-9 LD = 100E-9
+ CGDO = 200E-12 CGSO = 200E-12 CGBO = 1E-10
+ CJ = 400E-6 PB = 1 MJ = 0.5
+ CJSW = 300E-12 MJSW = 0.5
\end{verbatim}

So the initially designed pass transistor and its results were obtained as follows.
\begin{figure}[H]
	\centering
	\includegraphics[scale=0.5]{figures/2part2/without_res_cct.pdf}
	\caption{Schematic diagram of the first designed pass transistor}
\end{figure}
\begin{figure}[H]
	\centering
	\includegraphics[scale=0.5]{figures/2part2/without_res_wave.pdf}
	\caption{Waveform of the first designed pass transistor}
\end{figure}

But it was observed that there is a leakage voltage when the NMOS is at the high impedance state. So we used a resistor as a load to pull down the output to zero.

\begin{figure}[H]
	\centering
	\includegraphics[scale=0.5]{figures/2part2/small_res_cct.pdf}
	\caption{Schematic diagram of the second designed pass transistor}
\end{figure}
\begin{figure}[H]
	\centering
	\includegraphics[scale=0.5]{figures/2part2/without_res_wave.pdf}
	\caption{Waveform of the second designed pass transistor}
\end{figure}

As observed, the 1 k$\Omega$ load resistance was not enough to satisfy the necessity. So, we used a larger load instead.

\begin{figure}[H]
	\centering
	\includegraphics[scale=0.5]{figures/2part2/pass_ttr.pdf}
	\caption{Schematic diagram of the finalized pass transistor}
\end{figure}
\begin{figure}[H]
	\centering
	\includegraphics[scale=0.5]{figures/2part2/pass_wave.pdf}
	\caption{Waveform of the finalized pass transistor}
\end{figure}

Then using six such transistors, the single switch matrix was designed and the schematic diagram of it is shown below.

\begin{figure}[H]
	\centering
	\includegraphics[scale=0.5]{figures/2part2/switch_mat.pdf}
	\caption{Schematic diagram of the single switch matrix}
\end{figure}
\begin{figure}[H]
	\centering
	\includegraphics[scale=0.5]{figures/2part2/block.pdf}
	\caption{Designed single switch matrix block}
\end{figure}

Finally the functionality of the circuit was checked by giving pulses to left, right, top, and bottom corners separately and switching on the switches at different periods.

\begin{figure}[H]
	\centering
	\includegraphics[scale=0.5]{figures/2part2/top.pdf}
	\caption{Waveforms when the top terminal is fed with a pulse}
\end{figure}

\begin{figure}[H]
	\centering
	\includegraphics[scale=0.5]{figures/2part2/left.pdf}
	\caption{Waveforms when the left terminal is fed with a pulse}
\end{figure}

\begin{figure}[H]
	\centering
	\includegraphics[scale=0.5]{figures/2part2/right.pdf}
	\caption{Waveforms when the right terminal is fed with a pulse}
\end{figure}

\begin{figure}[H]
	\centering
	\includegraphics[scale=0.5]{figures/2part2/bottom.pdf}
	\caption{Waveforms when the bottom terminal is fed with a pulse}
\end{figure}


\pagebreak
\subsection{Part 3}
\textbf{Objective}: \textit{Design a PLD that can be used to design any 3 input
combinational circuit.}\\

The task was to design a PLD circuit capable of implementing any three input combinational circuit. The truth-table of any three input combinational circuits will be as below.\\

\begin{table}[H]
\centering
	\begin{tabular}{|c |c| c| c|}
		\hline
		A&B&C& Output\\\hline
	0 & 0 & 0 & $S_1 $ \\
	0 & 0 & 1 & $S_2$ \\
	0 & 1 & 0 & $S_3$ \\
	0 & 1 & 1 & $S_4$ \\
	1 & 0 & 0 & $S_5$ \\
	1 & 0 & 1 & $S_6$ \\
	1 & 1 & 0 & $S_7$ \\
	1 & 1 & 1 & $S_8$ \\\hline\hline
	\end{tabular}
\caption{The truth-table of any three input combinational circuit}
\end{table}

Outputs $S_1$, $S_2$,...., $S_8$ differ with the combinational circuit. So we can write an expression for the combinational logic circuit using the 8 minterms. Which minterms to be selected differ according to the S1, S2,...., S8. If any $S_i$ is 1 then the corresponding minterm is taken into the sum of products expression. If Si is 0 that corresponding minterm is discarded.\\

So we can build the PLD with a fixed AND plane which has all eight minterms and a programmable OR plane which can be programmed using Si terms. So our PLD becomes a PROM.\\

Before building the PLD the AND plane and OR plane should be created. For the fixed AND plane, we need eight minterms. A minterm is a product of any three of $A$, $\overline{A}$, $B$, $\overline{B}$, $C$ or $\overline{C}$. So we need three input and gate. We configured a three-input AND gate using NAND, NOR, and NOT gates as below for better efficiency.\\

\[ A.B.C = \overline{\overline{A.B.C}} = \overline{\overline{A.B} + \overline{C}} \]

Using this expression we constructed the 3 input AND gates using a minimum number of logic gates.\\

\begin{figure}[H]
	\centering
	\includegraphics[scale=0.7]{figures/Figure332.pdf}
	\caption{Implementing the three-input AND gate using NOR, NAND, and NOT gates.}
\end{figure}

Using seven separate OR gates( 7 NOR gates + 7 NOT gates) to implement the OR plane, increases complexity and the latency of the circuit by a huge factor. Instead, we can simplify the expression and use a minimum number of gates as below.\\

\[
\begin{split}
 & = S_1 + S_2 + S_3 + S_4 + S_5 + S_6 + S_7 + S_8\\
 & = \overline{\overline{S_1 + S_2 + S_3 + S_4 + S_5 + S_6 + S_7 + S_8}}\\
 & = \overline{\overline{\left( S_1 + S_2 + S_3 + S_4 \right)}. \overline{\left( S_5 + S_6 + S_7 + S_8 \right)}}\\
  & = \overline{\overline{\left( S_1 + S_2\right)} .\overline{\left( S_3 + S_4 \right)}. \overline{\left( S_5 + S_6\right)}.  \overline{\left( S_7 + S_8 \right)}}\\
    & = \overline{\overline{\left( S_1 + S_2\right)} .\overline{\left( S_3 + S_4 \right)}}+ \overline{\overline{\left( S_5 + S_6\right)}.  \overline{\left( S_7 + S_8 \right)}}\\
    & = \overline{\overline{ \overline{\overline{\left( S_1 + S_2\right)} .\overline{\left( S_3 + S_4 \right)}}+ \overline{\overline{\left( S_5 + S_6\right)}.  \overline{\left( S_7 + S_8 \right)}} }}
\end{split}
\]

Using this expression we were able to build an OR plane with a minimum number of components as below.\\

\begin{figure}[H]
	\centering
	\includegraphics[scale=0.5]{figures/Figure333.pdf}
	\caption{Implementing the OR plane}
\end{figure}


Instead of using a total of 14 logic gates, now we have implemented it using only 8 logic gates. This reduces the latency and complexity by a huge factor.\\

PROM is constructed as below\\

\begin{figure}[H]
	\centering
	\includegraphics[scale=0.5]{figures/Figure334.pdf}
	\caption{PROM circuit}
\end{figure}


We have used nmos transistors as switches which choose, which minterms are taken into the sum of products.\\

We tested the circuit for different combinational circuits by configuring Si switches. Below we have configured the PROM as a simple NOR gate.\\


\begin{figure}[H]
	\centering
	\includegraphics[scale=0.4]{figures/Figure335.pdf}
	\caption{PROM configured as a NOR gate}
\end{figure}

Results were as below,
\begin{figure}[H]
	\centering
	\includegraphics[scale=0.5]{figures/Figure336.pdf}
	\caption{Results of PROM configured as a NOR gate}
\end{figure}

We can observe that the PROM is functioning correctly.

\subsection{Discussion}


One major problem we encountered when designing the PLD was, configuring the programmability of the OR plane. For the programmability we need configurable switches. Each switch decides the corresponding min-term get added to the final expression or not. Expected characteristics of the switches are,
\begin{enumerate}[\hspace{1cm}1.]
	\item Giving the corresponding signal when the switch is on
	\item Giving binary 0 at the output, when the switch is off.

\end{enumerate}
There are two possible ways for achieving this.
\begin{enumerate}[\hspace{1cm}1.]
	\item Using a 2 input AND gate with corresponding signal and switch bit as two inputs.
	\item NMOS transistor used as a pass transistor with, switch bit given to the gate terminal and corresponding signal given to the source terminal.

\end{enumerate}
Considering the power efficiency and latencies we concluded that using NMOS as a pass transistor is the best way to achieve our goal.
To achieve the second characteristic we expected, giving binary 0 at the output, when the switch is off, we have connected a large load at the output (drain terminal) of the switch as pull
down resistor. So, when transistor is at the high impedance state it will be grounded through the resistor achieving binary 0.\\
\vfill

\hrule
{
\bibliographystyle{plain}
\bibliography{refer}
}
%---------------------------------------------------------------------------
\end{document}
